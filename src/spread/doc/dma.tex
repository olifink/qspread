\t Dynamic Memory Area description

\j The dynamic memory area (here: dma, not to be confused the method
of direct memory access) consists of one assembler file, that may,
in principle, be used by any other assembler program, provided that
this program follows some simple roules:

\f

        - a6 is a pointer to the data area of the job
        - da_fblok / da_lblok / da_ffree / da_lfree are all
          offsets in the data area (a6) capable of holding one
          long word each, allowing for run-time pointers

\j If these definitions can not be achieved by the application,
it should be no problem to change the sources accordingly. It is
vital that four pointers are provided for the use of the dma.


\t How to access the DMA

\j The application wanting to use dynamic memory has, at first,
to allocate one block of memory. This is all it has to deal with.
In principle, it can then allocate and free memory as much as it
likes. There is one routine to clean up the memory used by the
dma if needed, but it is also possbile to end the job via the
SMS.FRJB, which will release all memory allocated to this job.

\f

        \ob dma_frst \pn
        Allocate first block of memory

                Entry                   Exit
        a6      ptr to data area        preserved

        error codes: err.imem   not enough memory for first block
        condition codes set

\j This routine allocates the first memory block to be used by
the dma management routines. It sets up the block to the format
needed by the dma management and updates the long word pointers
in the data area.

\f

        dma_aloc
        Allocate space in the dma

                Entry                   Exit
        d1.l    nr. of bytes requested  preserved
        a0                              base of area allocated
        a6      ptr to data area        preserved

        error codes: err.imem   not enough system memory left
                     err.orng   req. size is bigger than one block
        condition codes set

\j This routine tries to allocate the requested space in the dynamic
memory area, and clears it. It is not possible to allocate more space
than slt.len*(blk.slts-1). In this case, err.orng will occur. This
is a problem the programmer has to think about and adjust the symbols
accordingly.

\f

        \ob dma_free \pn
        Relink unused space to the dma

                Entry                   Exit
        a0      base of area to free    preserved
        a6      ptr to data area        preserved

        error codes: none (d0.l unchanged)
        codition codes not set

\j This routine is assumed not to fail. It releases space previously
alloacted using the dma_aloc routine.

\f

        \ob dma_quit \pn
        Release the complete dynamic memory area

                Entry                   Exit
        a6      ptr to data area        preserved

        error codes: none (d0.l unchanged)
        condition codes not set

\j This routine releases all memory allocated to the dma. It is
the final step if the application wants to clean up its memory or
start new dynamic memory allocation. All runtime pointers area
reset.


\t Dynamic memory area concept

\j The concept of dma contains the defintion of blocks and slots.
Each block consists of a total of blk.slts slots. Each slots
has a size of slt.len bytes. Thus giving a size of blk.len=
blk.slts*slt.len bytes. A slot has to be at least $0c bytes long,
to hold the internal data structures. So, the allocation is done
in slots, i.e. even if only two bytes or so are required, one slot
is allocated.

There area two double linked lists in the dma. The one is holding
the blocks together, the other is keeping track of the free slots.
The first slot in each block is used to hold the pointers to the
following and the previous block, or the end of list marker
dma.eol for the last or first block respectivly. The remaining
slots of the block (blk.slts-1) are free for use. The block
list is held together by the long word pointers da_fblok and
da_lblok which point to the first and last block in list.
\f \mn


        *
        * Block desription
        blk_prev equ  $00       l ptr to previous block
        blk_next equ  $04       l ptr to next block
        blk.slts equ  $40         nr of slots in one block
        blk.len  equ  blk.slts * slt.len
        dma.eol  equ -$01         end of linked list marker


\j \pn Each slot starts with a word containing the total number of
free slots following (incl. this one). This is followed again
by two long word pointers to next and previous free slot accordingly.
Again, the last and the first slot have the end of list marker
at these positions. As the blocks, the free slots are held together
by the long word pointers da_ffree / da_lfree in the data area.

\f \mn


        *
        * Slot description
        slt_size equ  $00       w total number of free slots
        slt_prev equ  $02       l ptr to previous free slot
        slt_next equ  $06       l ptr to next free slot
        slt.len  equ  $10         length of one slot in bytes


\j \pn The allocation of space in the dma works as follows. First the
number of bytes to be allocated is increased by 2 bytes for internal
reasons explained later. From this, the number of slots required
is calculated. The management routine then enters the free slots
list via da_ffree and then checks the number of slots currently
available. There are three cases to be distinguished: firstly,
there are to few slots to satisfy the needs, where the routine
follows the list down to the next free slots. Secondly,
there are exactly the number of slots required, where the
previous and the following free slots are linked and the slot
found is prepared for use. The third case is, that there are more
slots available than actually required, where the previous slots
are linked to the slots behind the number of slots required.
If the allocation routine enounters the end of list pointer,
it tries to expand the dynamic memory area by another block (which
is linked in the block list, and than its free slots are linked to
the free slots list) and starts scanning there. The allocation routine
can therefore only allocate a total of blk.len-slt.len bytes.

The area found is then cleared, and the first word of it contains
the length of the area in slots. Then the address of the area +2 is
return to the application. The number of slots is required in the
dma_free routine, as it has to determine its size. Again, the area
is cleared and linked to the start of the free space list. This
ensures that the released free space is, if possible, used. This
is more efficient as it minimises fragmentation.

To clear the complete dynamic memory area, the management routine
follows the block list to the end, and releases each block to the
common heap again. Single block release is not supported, as it
increases heap fragmentation more than it would help to get free
space.
