\t Formular parser documentation

\j The formular parser provides a way to get values out of a
formular string, allowing for variables and user supplied functions.
The working of the parser is definied here, as well as the routines
is uses and the data structures it is working on.
In order to allow a rather high degree of flexibility, at a reasonable
amount of overhead code, the parser has to have access to the following
resourses, to be supplied by the calling program:

\f
     - pointer to formular string
     - pointer to result area
     - pointer to function table list
     - address of variable evaluation procedure

\j The parser itself has the following internal structures:

\f
     - table of functions
     - value stack
     - pending operations list
     - pending operations count



\t Working of the parser

\j The parser scans the formular string from the start, trying to
get one item. This item is then checked if it is a number, in which
case it can not start with character. For this decimal number, the
QDOS/SMS2 conversion routine is called to get a floating point value. This
may fail. The value, when found, gets onto the value stack.
The delimiting character of the item is then checked.
If it is end of line, the work is done, and the result is left on
top of the value (RI) stack. Otherwise the operations
code will be placed in the operation list and then pending operations
counter will be increased. The scan then continues for the next item.

There are some special cases, in which the parser has to react in a
slightly different way. If there are any immediate operators pending,
these will be carried out first, as their name implies.

The other kind special cases are the braces. These will, depending
on whether they were opening or closing, raise or lower the calculation
level the parser is working in, so that all expressions imbedded in
a pair or braces will be carried out first. Any non-matching number
of braces will lead to errors, too.



\t Formular string defintion

\j The formular string is a standard QDOS/SMS2 string. Beware: There
has to be space for one additional character, placed at the end of the
string! The parser tries to
find out the strings meaning (interpret it).
The string consists of one or more
items, which may be one of the following types:

\f
          fm.val    equ  $00  a floating point value
          fm.fn     equ  $02  a function name
          fm.var    equ  $04  a variable name


\j Each item of the string has to be delimited by a special character,
that has a meaning of its own to the parser. They are here
referred to as operations. The following standard operations are
supported, any others will lead to a parsing error. Imbedded spaces
are not treated as delimiters. There is another type of delimiters,
called control characters. These are the opening and closing braces,
as well as the end-of-function marker the parser itself places at its
end for simplicity reasons.

\f
          op.add    equ  '+'  ADD
          op.sub    equ  '-'  SUBstract
          op.mul    equ  '*'  MULtiply
          op.div    equ  '/'  DIVide
          op.pow    equ  '^'  power
          cc.open   equ  '('  opening braces
          cc.clos   equ  ')'  closing braces
          cc.end    equ  $00  end of function


\j By definition, all operations have to be followed by another item.
All control characters are treated a in a special way. Operaters and
control characters also invoke special internal routines and are, too,
items of its own.

There is one routine which allows for getting a standard item. It is
defined as follows:

\f

     Get Item from formular string
     prs_item

          Entry                         Exit
     A0   pointer to formular string    preserved
     A1   item buffer                   preserved
     D1.w parsing position              updated
     D2.b                               delimiter code

     D0 error code set


\j If an error occurs during scanning for this item, D1.w will be
updated to the position where the error occured.


\t Function table

\j A function in this parsing routine is identified by its name, followed
by an opening bracket. There are two types of functions, internal and
external. Internal functions are those originally supported by the
QDOS/SMS2 arithmetic package (like sin, cos, tan, exp etc.). All these
functions take one argument, that has to be enclosed in braces.
The internal function table looks like this

\f
         *                          internal function table layout
         ift_name equ      $00      s internal function name
         ift.end  equ      -1
         ift_qaop equ      $06      w qa operation code
         ift.len  equ      $08        length of one entry

         fn_intern
                  dc.w     3,'abs ',qa.abs
                  dc.w     3,'cos ',qa.cos
                  dc.w     3,'sin ',qa.sin
                  dc.w     3,'tan ',qa.tan
                  dc.w     3,'cot ',qa.cot
                  dc.w     4,'asin',qa.asin
                  dc.w     4,'acos',qa.acos
                  dc.w     4,'atan',qa.atan
                  dc.w     4,'acot',qa.acot
                  dc.w     4,'sqrt',qa.sqrt
                  dc.w     2,'ln  ',qa.ln
                  dc.w     3,'log ',qa.log10
                  dc.w     3,'exp ',qa.exp


\j The external functions table allows the programmer to easily extend
the list of functions for the special purposes of the application.
The function name is defined as a standard string in the external
function table, and may be up to six characters in length, which should
satisfy most needs.

The external function table should look like this:

\f
         *                          external function table layout
         xft_name equ      $00      s external function name
         xft.end  equ      -1         end of table marker
         xft_args equ      $08      w nr. of arguments
         xft_fn   equ      $0a      l rel. ptr to function code
         xft.len  equ      $0e


\j At present stage, the number of arguments is restriced to 1. If the
number of arguments is given as 0, the external code will be called in
a slightly different way: the expression (or whatever) given between
the braces is ignored by the parser and passed on to the external
function routine, which then has to deal with it.

The external routines are called by the parser with the
following arguments. A routine which expects an argument, will find it
on top of the stack. This can them be replaced by the value that
should be returned to the parser as the result. Other routines have
to adjust the stack and place the new value on top of it.
The routine has
to set the D0 register to the error code state, when it was called
with a value out of range or others that would lead to errors.

\f

     User supplied function code

          Entry                         Exit
     A0  (ptr to formular)              ???
     A1   top of stack                  updated
     D1  (length of arg.-string)        ???
     D2  (pos. of arg.-string)          ???

     Error codes: must be set!


\j As most of the external function have to carry out their own
arthmetic calculations, a simple way is provided to access the
arthmetic package:

\f

     Execute an arthmetic function
     (prs_doqa)

          Entry                         Exit
     A1   top stack                     updated
     D0   operation code                error (err.iexp)



\t Variable evaluation

\j In order to allow a high degree of flexibility
in variable defintion, such as
name and structure of the variable area, the parser has to have
access to a routine, that gets the actual value for a variable with
a given name. The only limitation is, that this name has to start
with a letter and may not contain any of the delimition characters
described above. The error code has to be set as well, if the given
variable name has no definied values.

\f

     Variable evaluation routine

          Entry                    Exit
     A1   address of variable name preserved (now result)
     D0                            error code
